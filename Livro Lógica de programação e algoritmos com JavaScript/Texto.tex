Lógica de programação envolve:
1- Compreender o que é pedido: leia e releia, compreenda o problema a ser solucionado, se não vocÊ ficará frustado. Então perca o tempo que precisar para entender o problema.
2- Realizar deduções na construção do programa, exemplo:

1- Caio está no Brasil ou na Argentina?
2- Caio não está na Argentina!
Considerando as afirmações:
3 - Caio está no Brasil!

1- A bicicleta está na rua ou na casa da avó?
2- A bicicleta não está na rua!
Considerando as afirmações:
3- A bicicleta está na avó!

Deduzir conclusões será vital para deduzir sobre quais controles devem ser utilizados para melhor solucionar um problema.

3- Enumerar as etapas a serem realizadas: algumas ações realizadas em um programa seguem uma lógica sequencial,
ou seja, um comando (ação) é realizado após o outro.
Algumas ações realizadas em um programa seguem uma lógica sequencial,
ou seja, um comando (ação) é realizado após o outro. Vamos continuar no
exemplo do carro na garagem. Para sair com o carro, é necessário:

1. Abrir a porta do carro.
2. Entrar no carro.
3. Ligar o carro.
4. Abrir o portão da garagem.
5. Engatar a marcha ré.
6. Sair com o carro da garagem.
7. Fechar o portão.
8. Engatar a primeira marcha.
9. Dirigir ao destino.

Algumas dessas ações poderiam exigir a criação de condições,por exemplo, o que deve ser feito se o carro não ligar?

4- Analisar outras possibilidades de solução: Tenho três amigos e quero saber quem faz aniversário mais perto da data atual.
Como resolver esse problema?

1. Descobrir a maior idade.
2. Descobrir a segunda maior idade.
3. Somar as duas idades maiores.
OU
1. Descobrir quem tem a menor idade.
2. Somar a idade dos outros dois.

Quando deparar com um problema que você está com dificuldades
para resolver de uma forma, respire um pouco... Tome uma água... Tente
pensar se poderia existir outra forma de solucioná-lo.

5- Ensinar ao computador uma solução: Um exemplo simples: você precisa calcular o número total de horas de uma
viagem, expressa em dias e horas. Uma viagem para Florianópolis dura 2 dias e 5 horas, por exemplo.
Qual é a duração total dessa viagem em número de horas?
48+5= 53 horas.
Na resolução de um algoritmo, é necessário ensinar ao computador quais operações devem ser realizadas para se chegar a uma solução correta
para o problema. Ou seja, deve-se primeiro entender como solucionar o problema para depois passá-lo para o algoritmo.

6- Pensar em todos os detalhes: comparação de uma receita de bolo:
Na montagem de uma receita de bolo, temos os ingredientes (como os dados de entrada), as ações a serem realizadas sobre os
ingredientes (processamento) e o resultado esperado, que é o bolo em si (como os dados de saída). Esquecer algum ingrediente
ou detalhe de alguma ação certamente fará com que o bolo não fique conforme o planejado.

1. Pegar uma caixa de fósforo.
2. Abrir a caixa de fósforo.
3. Verificar se tem palito. Se Sim:
3.1 Retirar um palito.
3.2 Fechar a caixa.
3.3 Riscar o palito.
3.4 Verificar se acendeu. Se Sim:
3.4.1 Ok! Processo Concluído.
3.5 Se não: Retornar ao passo? // Qual passo? Se não especificarmos qual passo o código começa a ser infinito.
4. Se não: Descartar a caixa e retornar ao passo 1.

Comandos sequencias: uma ação realizada após a outra como 1-, 2-, etc.
Comandos para definição de condições: servem para determinar quais comandos serão executados a partir da análise de uma condiçao.
Se  uma condição retornar verdadeiro, o programa segue por um caminho, se falso, por outro. (3. Verificar se tem palito? ou 3.4, Verificar se
acendeu?)
Estruturas de repetições: indicam que uma ação ou conjunto de ações devem ocorrer várias vezes  (retornar ao passo 2 e retornar ao passo
1). Nessas estruturas, é preciso indicar quantas vezes a repetição vai ocorrer ou criar algum ponto de saída no laço.

7- Entrada, processamento e saída: memorizar o roteiro de etapas a serem seguidas abaixo:

a) Leia os dados de entrada (exige soliticar ao usuário alguma informação, por exemplo, nome, idade, salaário, etc).
b) Realize o processamento dos dados (calcula o salário, calcula um desconto ou verifica a idade).
c) Apresente a saída dos dados. (exibição do novo salário, do desconto ou se a pessoa é maior ou menor de idade).

Os termos comando, função, método ou
procedimento podem ter pequenas diferenças quanto ao conceito, dependendo da linguagem.

1.6
const: conteúdo da variável não pode ser alterada.
let: conteúdo da variável pode ser alterada.

1.7- Entrada de dados com prompt()

1.8- Comentários são essenciais para um código numa futura manutenção
ou até mesmo para o programador continuar com a linha de raciocínio.

1.9- Tipos de dados e conversões de tipos.
Strings (variáveis de texto).
Números e valores boleanos (true ou false)
Converver números se utiliza NUMBER.
Converver um número em string se utiliza o método toString().
As entradas de dados realizadas com o método prompt()
criam variáveis do tipo String, exceto se houver uma função de conversão de
dados como Number(). Exibir uma variável que não recebeu uma atribuição
de valor vai gerar uma saída “undefined”.

1.10- 
